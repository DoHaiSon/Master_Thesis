%%% List of new cmds:


% Line
\tikzset{
    line/.style={draw, thick},
}

% Node S
\newcommand{\NodeS}[5]{\NodeSS{3mm}{draw=black, line width=1pt, fill=teal!30!white}{#4}{#5};
\node (#1) at (#2, #3) [node S, draw=black, fill=teal!10!white] {}
}
\newcommand{\NodeSS}[4]{
    \tikzset{
        old inner xsep/.estore in=\oldinnerxsep,
        old inner ysep/.estore in=\oldinnerysep,
        node S/.style 2 args={
            circle,
            old inner xsep=\pgfkeysvalueof{/pgf/inner xsep},
            old inner ysep=\pgfkeysvalueof{/pgf/inner ysep},
            /pgf/inner xsep=\oldinnerxsep+#1,
            /pgf/inner ysep=\oldinnerysep+#1,
            alias=sourcenode,
            append after command={
            let     \p1 = (sourcenode.center),
                    \p2 = (sourcenode.east),
                    \n1 = {\x2-\x1-#1-0.5*\pgflinewidth}
            in
                node [inner sep=#3, draw, fill, circle, minimum width=2*\n1,at=(\p1), #2] {#4}     % Outside box
            }
        },
        node S/.default={3mm}{black}     % inside  
    }
}

% Node R
\newcommand{\NodeR}[5]{\NodeRR{3mm}{draw=black, line width=1pt, fill=green!30!white}{#4}{#5};
\node (#1) at (#2, #3) [node R, draw=black, fill=green!10!white] {}
}
\newcommand{\NodeRR}[4]{
    \tikzset{
        old inner xsep/.estore in=\oldinnerxsep,
        old inner ysep/.estore in=\oldinnerysep,
        node R/.style 2 args={
            circle,
            old inner xsep=\pgfkeysvalueof{/pgf/inner xsep},
            old inner ysep=\pgfkeysvalueof{/pgf/inner ysep},
            /pgf/inner xsep=\oldinnerxsep+#1,
            /pgf/inner ysep=\oldinnerysep+#1,
            alias=sourcenode,
            append after command={
            let     \p1 = (sourcenode.center),
                    \p2 = (sourcenode.east),
                    \n1 = {\x2-\x1-#1-0.5*\pgflinewidth}
            in
                node [inner sep=#3, draw, fill, circle, minimum width=2*\n1,at=(\p1), #2] {#4}     % Outside box
            }
        },
        node R/.default={3mm}{black}     % inside  
    }
}

% Node E
\newcommand{\NodeE}[5]{\NodeEE{3mm}{draw=black, line width=1pt, fill=red!30!white}{#4}{#5};
\node (#1) at (#2, #3) [node E, draw=black, fill=red!10!white] {}
}
\newcommand{\NodeEE}[4]{
    \tikzset{
        old inner xsep/.estore in=\oldinnerxsep,
        old inner ysep/.estore in=\oldinnerysep,
        node E/.style 2 args={
            circle,
            old inner xsep=\pgfkeysvalueof{/pgf/inner xsep},
            old inner ysep=\pgfkeysvalueof{/pgf/inner ysep},
            /pgf/inner xsep=\oldinnerxsep+#1,
            /pgf/inner ysep=\oldinnerysep+#1,
            alias=sourcenode,
            append after command={
            let     \p1 = (sourcenode.center),
                    \p2 = (sourcenode.east),
                    \n1 = {\x2-\x1-#1-0.5*\pgflinewidth}
            in
                node [inner sep=#3, draw, fill, circle, minimum width=2*\n1,at=(\p1), #2] {#4}     % Outside box
            }
        },
        node E/.default={3mm}{black, line width=1pt, fill=red!30!white}     % inside  
    }
}

% Node D
\newcommand{\NodeD}[5]{\NodeDD{3mm}{draw=black, line width=1pt, fill=blue!30!white}{#4}{#5};
\node (#1) at (#2, #3) [node D, draw=black, fill=blue!10!white] {}
}
\newcommand{\NodeDD}[4]{
    \tikzset{
        old inner xsep/.estore in=\oldinnerxsep,
        old inner ysep/.estore in=\oldinnerysep,
        node D/.style 2 args={
            circle,
            old inner xsep=\pgfkeysvalueof{/pgf/inner xsep},
            old inner ysep=\pgfkeysvalueof{/pgf/inner ysep},
            /pgf/inner xsep=\oldinnerxsep+#1,
            /pgf/inner ysep=\oldinnerysep+#1,
            alias=sourcenode,
            append after command={
            let     \p1 = (sourcenode.center),
                    \p2 = (sourcenode.east),
                    \n1 = {\x2-\x1-#1-0.5*\pgflinewidth}
            in
                node [inner sep=#3, draw, fill, circle, minimum width=2*\n1,at=(\p1), #2] {#4}     % Outside box
            }
        },
        node D/.default={3mm}{black, line width=1pt, fill=blue!30!white}     % inside 
    }
}

% XOR symbol
\tikzset{XOR/.style={draw,circle,append after command={
        [shorten >=\pgflinewidth, shorten <=\pgflinewidth,]
        (\tikzlastnode.north) edge (\tikzlastnode.south)
        (\tikzlastnode.east) edge (\tikzlastnode.west)
        }
    }
}


% circle dot line
\tikzset{decorate sep/.style 2 args=
{decorate,decoration={shape backgrounds,shape=circle,shape size=#1,shape sep=#2}}}

% NNN
\newcommand{\NNN}{
    \newcommand{\inputnum}{4} 
    \newcommand{\numhiddenlayer}{2}
    \newcommand{\hiddennum}{5}  
    \newcommand{\outputnum}{1} 
    
    \begin{tikzpicture}
        % Input Layer
        \foreach \i in {1,...,\inputnum}
        {
            \node[circle, 
                minimum size = 6mm,
                fill=orange!30] (Input-\i) at (0,-\i) {};
        }
        \foreach \i in {1,...,3}
        {
            \draw node [circle, fill=orange!30, scale=0.5] at (0, -\i - 0.5) {};
        }
         
        % Hidden Layer
        \foreach \i in {1,...,\numhiddenlayer}
        {
        \foreach \j in {1,...,\hiddennum}
            {
                \node[circle, 
                    minimum size = 6mm,
                    fill=teal!50,
                    yshift=(\hiddennum-\inputnum)*5 mm
                ] (Hidden-\i-\j) at (2.5 * \i,-\j) {};
            }
        \foreach \j in {1,...,4}
            {
                \draw node [circle, fill=teal!50, scale=0.5] at (2.5 * \i, -\j) {};
            }    
        }
        
        % Output Layer
        \foreach \i in {1,...,\outputnum}
        {
            \node[circle, 
                minimum size = 6mm,
                fill=purple!50,
                yshift=(\outputnum-\inputnum)*5 mm
            ] (Output-\i) at (7,-\i) {};
        }
         
        % Connect neurons In-Hidden
        \foreach \i in {1,...,\inputnum}
        {
            \foreach \j in {1,...,\hiddennum}
            {
                \draw[->, shorten >=1pt, -stealth] (Input-\i) -- (Hidden-1-\j);   
            }
        }
         
        % Connect neurons Hidden-Hidden
        \foreach \i in {1,...,\hiddennum}
        {
            \foreach \j in {1,...,\hiddennum}
            {
                \draw[->, shorten >=1pt, -stealth] (Hidden-1-\i) -- (Hidden-2-\j);   
            }
        }
         
        % Connect neurons Hidden-Out
        \foreach \i in {1,...,\hiddennum}
        {
            \foreach \j in {1,...,\outputnum}
            {
                \draw[->, shorten >=1pt, -stealth] (Hidden-\numhiddenlayer-\i) -- (Output-\j);
            }
        }
        
        % Inputs
        \foreach \i in {2,...,2}
        {            
            \draw[<-, line width = 2pt, stealth-] ([xshift=-10pt]Input-\i.west) -- ++(-1,0)
                node[left, label=$X_n$]{};
        }
        \foreach \i in {3,...,3}
        {            
            \draw[<-, line width = 2pt, stealth-] ([xshift=-10pt]Input-\i.west) -- ++(-1,0)
                node[left, label=$Y_n / Z_n$]{};
        }
         
        % Outputs
        \foreach \i in {1,...,\outputnum}
        {            
            \draw[<-, line width = 2pt, -stealth] (Output-\i) -- ++(2,0)
                node[pos=0.3,right, label={$T(x_k, y_k / z_k)$}]{};
        }
    \end{tikzpicture}
}

\tikzstyle{startstop} = [rectangle, rounded corners, minimum width=3cm, minimum height=1cm,text centered, draw=black, fill=red!10!white]

\tikzstyle{io} = [trapezium, trapezium left angle=70, trapezium right angle=110, minimum width=3cm, minimum height=1cm, text centered, draw=black, fill=orange!10!white]

\tikzstyle{process} = [rectangle, minimum width=3cm, minimum height=1cm, text centered, draw=black, fill=blue!10!white]

\tikzstyle{decision} = [diamond, minimum width=3cm, minimum height=1cm, text centered, draw=black, fill=green!10!white]

\tikzstyle{arrow} = [thick,->,>=stealth]

\tikzstyle{darrow} = [thick, stealth-stealth]

\tikzstyle{arrowt1}  = [thick,->,>=stealth, draw=red]

\tikzstyle{arrowt2}  = [thick,->,>=stealth, dashed, draw=green]

\tikzstyle{arrowt3}  = [thick,->,>=stealth, dotted, draw=black]

\tikzstyle{field} = [rectangle, minimum width=5mm, minimum height=10mm, align=center, fill=green!10!white, font=\footnotesize, draw=black, line width=1pt]

\tikzstyle{field1} = [rectangle, minimum width=20mm, minimum height=5mm, align=center, fill=green!10!white, font=\scriptsize, draw=black, line width=1pt, rotate=-90]

\tikzstyle{field2} = [rectangle, minimum width=5mm, minimum height=20mm, align=center, fill=green!10!white, font=\scriptsize, draw=black, line width=1pt]

\tikzstyle{field3} = [rectangle, minimum width=31mm, minimum height=5mm, align=center, fill=green!10!white, font=\small, draw=black, line width=1pt, rotate=-90]

\tikzstyle{field4} = [rectangle, minimum width=31mm, minimum height=5mm, align=center, fill=green!10!white, draw=black, line width=1pt, rotate=-90]

\tikzstyle{field5} = [rectangle, minimum width=3mm, minimum height=10mm, align=center, fill=green!10!white, draw=black, line width=1pt]

\tikzstyle{field6} = [rectangle, minimum width=15mm, minimum height=10mm, align=center, fill=green!10!white, draw=black, line width=1pt]

\tikzstyle{field7} = [rectangle, minimum width=3mm, minimum height=10mm, align=center, fill=green!10!white, font=\footnotesize, draw=black, line width=1pt]

\tikzstyle{field8} = [rounded rectangle, minimum width=10mm, minimum height=10mm, align=center, fill=green!10!white, draw=black, line width=1pt]

\makeatletter
\tikzset{
    database/.style={
        path picture={
            \draw (0, 1.5*\database@segmentheight) circle [x radius=\database@radius,y radius=\database@aspectratio*\database@radius];
            \draw (-\database@radius, 0.5*\database@segmentheight) arc [start angle=180,end angle=360,x radius=\database@radius, y radius=\database@aspectratio*\database@radius];
            \draw (-\database@radius,-0.5*\database@segmentheight) arc [start angle=180,end angle=360,x radius=\database@radius, y radius=\database@aspectratio*\database@radius];
            \draw (-\database@radius,1.5*\database@segmentheight) -- ++(0,-3*\database@segmentheight) arc [start angle=180,end angle=360,x radius=\database@radius, y radius=\database@aspectratio*\database@radius] -- ++(0,3*\database@segmentheight);
        },
        minimum width=2*\database@radius + \pgflinewidth,
        minimum height=3*\database@segmentheight + 2*\database@aspectratio*\database@radius + \pgflinewidth,
    },
    database segment height/.store in=\database@segmentheight,
    database radius/.store in=\database@radius,
    database aspect ratio/.store in=\database@aspectratio,
    database segment height=0.1cm,
    database radius=0.25cm,
    database aspect ratio=0.35,
}
\makeatother