\clearpage
\phantomsection

\addcontentsline{toc}{chapter}{Danh mục ký hiệu và chữ viết tắt}
\chapter*{Danh mục ký hiệu và chữ viết tắt}
{\renewcommand{\arraystretch}{1.4}
{\fontsize{12}{13}\selectfont
\begin{longtable}{|>{\raggedright}p{4.4cm}|p{10.4cm}|}
\hline
\multicolumn{2}{|l|}{\textbf{Danh mục ký hiệu}} \\ 
\hline
\hline
\textbf{Ký hiệu} & ~\textbf{Giải thích}~ \\ 
\hline
in thường & Vô hướng \\ 
\hline
in thường, đậm & Véc-tơ \\ 
\hline
in hoa, đậm & Ma trận \\ 
\hline
(.)$^\top$ & Chuyển vị \\ 
\hline
(.)$^{-1}$ & Nghịch đảo \\ 
\hline
(.)$^H$ & Phép biến đổi Hermitian \\ 
\hline
$\alpha$ & Tham số học của mạng ISDNN \\ 
\hline
$\delta$ & Tốc độ học / độ dài bước \\ 
\hline
$\Im$ & Phần ảo \\ 
\hline
$\lambda$ & Bước sóng \\ 
\hline
$\mathbb{E}$(.) & Kỳ vọng \\ 
\hline
$\mathbf{e}$ & Véc-tơ phần dư / lỗi \\ 
\hline
$\mathcal{L}$ & Hàm mất mát \\ 
\hline
$\mathcal{O}$ & Độ phức tạp \\ 
\hline
$\mathcal{U}$ & Phân bố đều \\ 
\hline
$\Psi$ & Toán tử phi tuyến tính \\ 
\hline
$\Psi_t$ & Hàm phi tuyến tính phân đoạn \\ 
\hline
$\Re$ & Phần thực \\ 
\hline
$\sigma$ & Độ lệch chuẩn \\ 
\hline
$\mathbf{\theta}$ & Góc ngẩng \\ 
\hline
$\mathcal{F}$ & Ma trận Fourier rời rạc \\ 
\hline
$\mathbf{0}_T$ & Véc-tơ cột gồm các phần tử $0$ có kích thước $T \times 1$ \\ 
\hline
$\operatorname{arg}$ & Vị trí của phần tử trong véc-tơ / ma trận \\ 
\hline
$b$ & Độ lệch của bộ biến đổi tuyến tính \\ 
\hline
$\mathcal{CN}$ & Phân phối chuẩn phức \\ 
\hline
$\operatorname{con}$ & Phép nối véc-tơ\\ 
\hline
$\partial$ & Đạo hàm riêng \\ 
\hline
dB & Đơn vị decibel \\ 
\hline
$\operatorname{diag}$ & Thành phần đường chéo \\ 
\hline
$\mathbf{G}$ & Ma trận của bộ nhận dạng tuyến tính \\ 
\hline
$\mathbf{G}_H$ & Ma trận Gram của $\mathbf{H}$ \\ 
\hline
$\mathbf{H}$ & Ma trận kênh truyền \\ 
\hline
$\mathbf{I}_K$ & Ma trận đơn vị kích thước $K \times K$ \\ 
\hline
$K$ & Độ dài một ký hiệu OFDM / Số các lớp trong một mạng DNN \\ 
\hline
$L$ & Số ăng-ten thu \\ 
\hline
$M$ & Số lượng đường truyền giữa một cặp ăng-ten thu phát \\ 
\hline
$\operatorname{min}$ & Giá trị nhỏ nhất \\ 
\hline
$n$ & Thời điểm \\ 
\hline
$K_p$ & Số lượng ký hiệu pilot \\ 
\hline
$K_d$ & Số lượng ký hiệu dữ liệu \\ 
\hline
$\rho$ & Toán tử tuyến tính \\ 
\hline
$\mathbf{s}$ & Các ký hiệu được gửi đi từ bộ phát \\ 
\hline
$T$ & Số ăng-ten phát / người dùng bên phát \\ 
\hline
$\operatorname{vec}(\mathbf{X})$ & Véc-tơ hoá ma trận $\mathbf{X}$ \\ 
\hline
$\mathbf{w}$ & Tạp âm AWGN \\ 
\hline
$w$ & Trọng số của bộ biến đổi tuyến tính \\ 
\hline
$\mathbf{x}$ & Các ký hiệu bên thu nhận được \\ 
\hline
$\Theta$ & Véc-tơ các tham số không biết trước cần được ước lượng / bộ tham số của việc học \\ 
\hline
$\beta$ & Hệ số khuếch đại \\ 
\hline
$\phi$ & Góc phương vị \\
\hline
\end{longtable}
}}
\newpage

{\renewcommand{\arraystretch}{1.2}
{\fontsize{12}{13}\selectfont
\begin{longtable}{|p{2.35cm}|>{\raggedright}p{6.2cm}|p{5.75cm}|}
\hline
\multicolumn{3}{|l|}{\textbf{Danh mục chữ viết tắt}} \\ 
\hline
\hline
\textbf{Chữ viết tắt}~ & \textbf{Giải thích tiếng Anh} & \textbf{Giải thích tiếng Việt} \\ 
\hline
AI & Artificial Intelligence & Trí tuệ nhân tạo \\ 
\hline
AWGN & Additive White Gaussian Noise & Tạp âm trắng cộng tính \\ 
\hline
B & Blind & Kỹ thuật nhận dạng mù \\ 
\hline
BCCH & Broadcast Control Channel & Kênh điều khiển quảng bá \\ 
\hline
BER & Bit Error Rate & Tỷ lệ lỗi bít \\ 
\hline
CBSM & Correlation-based Stochastic Mode & Mô hình ngẫu nhiên dựa trên tương quan\\
\hline
CCCH & Common Control Channel & Kênh điều khiển chung \\ 
\hline
CMA & Constant Modulus Algorithm & Thuật toán mô-đun không đổi \\ 
\hline
CP & Cyclic Prefix & Tiền tố vòng \\ 
\hline
CRB & Cramér Rao Bound & Đường bao Cramér Rao \\ 
\hline
CSI & Channel State Information & Thông tin về trạng thái kênh truyền \\ 
\hline
CSI-RS & Channel State Information - Reference Signal & Tín hiệu tham chiếu thông tin trạng thái kênh truyền \\ 
\hline
Data-driven & Data-driven & Hướng dữ liệu \\ 
\hline
DCCH & Dedicated Control Channel & Kênh điều khiển chuyên dụng \\ 
\hline
DDCE & Decision-directed Channel Estimation & Ước lượng kênh trực tiếp quyết định \\ 
\hline
DetNet & Detection Network & Mạng nơ-ron học sâu phát hiện \\ 
\hline
DL & Deep Learning & Học sâu \\ 
\hline
DL-SCH & Downlink Shared Channel & Kênh chia sẻ đường xuống \\ 
\hline
DMRS & Demodulation Reference Signal & Tín hiệu tham chiếu giải điều chế \\ 
\hline
DNN & Deep-neural Network & Mạng nơ-ron sâu \\ 
\hline
DoA & Direction of Arrival & Hướng sóng đến \\ 
\hline
DoD & Direction of Departure & Hướng phát sóng \\ 
\hline
DTCH & Dedicated Traffic Channel & Kênh lưu lượng chuyên dụng \\ 
\hline
FFT & Fast Fourier Transform & Biến đổi Fourier nhanh \\ 
\hline
FIM & Fisher Information Matrix & Ma trận thông tin Fisher \\ 
\hline
GBSM & Geometry-based Stochastic Model & Mô hình ngẫu nhiên dựa trên hình học\\
\hline
GPR & Gaussian Process Regression & Thuật toán hồi quy Gaussian \\ 
\hline
HOS & Higer-order Statistics & Đặc tính thống kê bậc cao \\ 
\hline
i.i.d & Independent and Identically Distributed & Biến độc lập và phân phối đồng nhất \\ 
\hline
ICA & Independent Component Analysis & Phân tích thành phần độc lập \\ 
\hline
ISD & Iterative Sequential Detection & Mạng học sâu phát hiện tuần tự lặp lại \\ 
\hline
ISDNN & Iterative Sequential Deep-neural Network & Mạng nơ-ron sâu tuần tự lặp lại \\ 
\hline
KBSM & Kronecker-based Stochastic Mode & Mô hình ngẫu nhiên Kronecker\\
\hline
LMMSE & Linear Minimum Mean Square Error & Kỹ thuật ước lượng lỗi bình phương trung bình tối thiểu tuyến tính \\ 
\hline
LMS & Least Mean Squares & Kỹ thuật trung bình bình phương tối thiểu \\ 
\hline
LS & Least Square & Kỹ thuật bình phương tối thiểu \\ 
\hline
LSTM & Long/Short-term Memory & Mạng trí nhớ dài hạn/ngắn hạn \\ 
\hline
MIMO & Multi-Input Multi-Output & Hệ thống đa đầu vào đa đầu ra \\ 
\hline
ML & Machine Learning & Học máy \\ 
\hline
MLE & Maximum Likelihood Estimator & Bộ ước lượng hợp lẽ cực đại \\ 
\hline
mMIMO & Massive Multi-Input Multi-Output & Hệ thống đa đầu vào đa đầu ra kích thước lớn \\ 
\hline
MMSE & Minimum Mean Square Error & Kỹ thuật ước lượng lỗi bình phương trung bình tối thiểu \\ 
\hline
mmWave & Millimeter Wave & Bước sóng mi-li-mét \\ 
\hline
Model-driven & Model-driven & Hướng mô hình \\ 
\hline
MRE & Mutually Referenced Equalizers & Bộ cân bằng kênh tham chiếu \\ 
\hline
NB & Non-Blind & Kỹ thuật nhận dạng không mù \\ 
\hline
NGSM & Non-Geometrical Stochastic Model & Mô hình kênh ngẫu nhiên không dựa trên hình học \\
\hline
NN & Neural Network & Mạng nơ-ron \\ 
\hline
OFDM & Orthogonal Frequency-division Multiplexing & Ghép kênh phân chia theo tần số trực giao \\ 
\hline
OP & Only Pilot & Chỉ sử dụng các ký hiệu hoa tiêu \\ 
\hline
PC & Pilot Contamination & Ô nhiễm pilot \\ 
\hline
PCA & Principal Components Analysis & Phân tích thành phần chính \\ 
\hline
PCCH & Paging Control Channel & Kênh điều khiển paging \\ 
\hline
PDF & Probability Density Function & Hàm mật độ xác suất \\ 
\hline
Pilot & Pilot & Ký hiệu hoa tiêu \\ 
\hline
Pilot-assisted & Pilot-assisted & Phương pháp dựa trên ký hiệu hoa tiêu \\ 
\hline
PSK & Phase-shift Keying & Điều chế pha số \\ 
\hline
PSS & Primary Synchronization Signal & Tín hiệu đồng bộ sơ cấp \\ 
\hline
PTRS & Phase Tracking Reference Signal & Tín hiệu tham chiếu bám pha \\ 
\hline
QAM & Quadrature Amplitude Modulation & Điều chế biên độ cầu phương \\ 
\hline
RACH & Random Access Channel & Kênh truy cập ngẫu nhiên \\ 
\hline
RL & Reinforcement Learning & Học tăng cường \\ 
\hline
SB & Semi-Blind & Kỹ thuật nhận dạng bán mù \\ 
\hline
SIMO & Single-Input Multi-Output & Hệ thống đơn đầu vào đa đầu ra \\ 
\hline
SISO & Single-Input Single-Output & Hệ thống đơn đầu đơn vào đầu ra \\ 
\hline
SNR & Signal Noise Ratio & Tỷ lệ công suất tín hiệu trên công suất tạp âm \\ 
\hline
SOS & Second-order Statistics & Đặc tính thống kê bậc hai \\ 
\hline
SRS & Sounding Reference Signal & Tín hiệu tham chiếu âm thanh \\ 
\hline
SSS & Secondary Synchronization Signal & Tín hiệu đồng bộ thứ cấp \\ 
% \hline
% SC & Sub-cluster & Cụm con trong các cụm phân chia theo thời gian đến \\
\hline
TC & Time Cluster & Cụm phân chia theo thời gian đến \\
% \hline
% Training-based & Training-based & Phương pháp dựa trên việc huấn luyện \\ 
% \hline
% RT  & Ray-Tracing & Kỹ thuật bám tia \\
\hline
UCA & Uniform Circle Array & Mảng tròn cách đều \\ 
\hline
UCyA & Uniform Cylindrical Array & Mảng trụ cách đều \\ 
\hline
UE & User Equipment & Thiết bị đầu cuối của người dùng \\
\hline
ULA & Uniform Linear Array & Mảng thẳng cách đều \\ 
\hline
UL-SCH & Uplink Shared Channel & Kênh chia sẻ đường lên \\ 
\hline
ZF & Zero Forcing & Kỹ thuật ép không \\
\hline
\end{longtable}
}
}   
			