\clearpage
\phantomsection

\addcontentsline{toc}{chapter}{Tóm tắt}
\chapter*{\fontsize{13}{13}\selectfont{Tóm tắt}}
\fontsize{12}{12}\selectfont{
\noindent\textbf{Tóm tắt:} Các thế hệ mạng di động như 5G hay WiFi $802.11$ax hiện nay đang phải sử dụng một phần đáng kể băng thông và tài nguyên tính toán cho việc nhận dạng kênh truyền vô tuyến. Luận văn tập trung vào giảm thiểu chi phí và độ phức tạp của việc ước lượng kênh truyền trong các hệ thống MIMO kích thước lớn bằng việc sử dụng các ``tri thức mới'', ví dụ phương pháp bán mù sử dụng thêm các thông tin bên lề về cấu hình mảng ăng-ten hay học máy, học sâu sử dụng thông tin khác với thông tin kênh. Trước hết, đường bao Cramér Rao được sử dụng để xem xét sai số ước lượng tối thiểu của việc nhận dạng kênh truyền khi thay đổi cấu hình của các mảng ăng-ten MIMO kích thước lớn cũng như khi sử dụng giải thuật bán mù nhằm giảm thiểu chi phí. Kết quả mô phỏng chỉ ra rằng các cấu hình mảng ăng-ten 3D (ví dụ: UCyA) không những giúp tiết kiệm diện tích triển khai mà còn giảm sai số trong quá trình ước lượng kênh truyền. Bên cạnh đó, sử dụng một phần thông tin thống kê từ dữ liệu chưa biết trước theo hướng tiếp cận bán mù cũng có thể làm tăng tính chính xác của việc ước lượng. Sau đó, một mạng nơ-ron sâu có tên ISDNN được đề xuất cho cả hai mô hình kênh truyền có cấu trúc và không sử dụng cấu trúc để nhận dạng các hệ thống viễn thông MIMO kích thước lớn. Mạng nơ-ron được đề xuất cho sai số và độ phức tạp đều thấp hơn các giải thuật ước lượng tuyến tính như ZF hay MMSE với độ lợi có thể lên đến $10^3$ về tỷ lệ lỗi bít và giảm $\mathcal{O}(L)$ về độ phức tạp. Ngoài ra, khi so sánh với một mạng nơ-ron sâu khác là DetNet, chỉ với $24$ tham số học, độ chính xác của ISDNN cũng vượt trội, tiệm cận tỷ lệ lỗi bít $10^{-4}$. Các nghiên cứu trong luận văn chỉ ra tiềm năng của việc ứng dụng các ``tri thức mới'' trong việc ước lượng kênh truyền, khi có thể giải quyết nhiều bài toán như chi phí, độ phức tạp, và hiệu năng. 

\vspace{0.5cm}
\noindent\textit{\textbf{Từ khóa:}} \textit{Nhận dạng hệ thống, MIMO kích thước lớn, tri thức mới, CRB, học sâu.}
}