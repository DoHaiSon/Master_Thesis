\clearpage
\phantomsection

\addcontentsline{toc}{chapter}{KẾT LUẬN}
\chapter*{Kết luận}

Trong luận văn, tác giả tập trung giải quyết các thách thức về chi phí và độ phức tạp của các phương pháp nhận dạng trong các thế hệ mạng di động mới sử dụng mMIMO. Trước hết, các phương pháp mô hình kênh truyền trong mMIMO được khảo sát để chọn ra phương pháp phù hợp cho nghiên cứu trong luận văn. Tiếp đến,
một khảo sát về bốn phương pháp nhận dạng hệ thống viễn thông không dây được trình bày. Qua đó, chỉ ra được sự cần thiết của việc ứng dụng tri thức mới vào bài toán nhận dạng hệ thống thông qua hai hướng tiếp cận là bán mù và sử dụng học sâu. 
Từ đó, trong chương~\ref{sec:CRB}, tác giả xem xét sự ảnh hưởng của các cấu hình mảng ăng-ten khác nhau và giải thuật SB đến độ chính xác của việc ước lượng kênh truyền dựa trên CRB. Kết quả chỉ ra rằng việc mô hình kênh truyền có cấu trúc, sử dụng các mảng ăng-ten 3D như UCyA, hay phương pháp SB đều có thể giúp giảm sai số của việc ước lượng kênh truyền đi đáng kể. Ngoài ra việc sử dụng các mảng ăng-ten 3D sẽ giúp tiết kiệm diện tích lắp đặt cho các trạm cơ sở đi đáng kể khi so sánh với các cấu hình ULA truyền thống.
Trong chương~\ref{sec:ML}, hướng tiếp cận học sâu, sử dụng thêm thông tin bên lề về hướng sóng đến và cấu hình mảng ăng-ten, cũng được tác giả xem xét để nhận dạng kênh truyền cho hệ thống mMIMO. Cụ thể, một mạng ISDNN được đề xuất nhằm giảm thiểu độ phức tạp và chi phí so với thuật toán ISD gốc dựa trên bộ nhận dạng MMSE. Kiến trúc mạng nơ-ron sâu được đề xuất chỉ yêu cầu $24$ tham số học và $7$~KB cho mô hình được đào tạo với cấu hình gồm $4$ lớp mạng. Đây là số lượng rất nhỏ và hoàn toàn vượt trội khi so sánh với một mạng nơ-ron sâu khác cũng với cách tiếp cận tương tự là DetNet. Từ các kết quả mô phỏng, hiệu suất về thời gian đào tạo và độ chính xác của ISDNN cũng được kiểm chứng là vượt trội cả các phương pháp tuyến tính và mạng nơ-ron sâu DetNet.
Ngoài ra, tác giả cũng xem xét đến hiệu suất của mạng nếu dữ liệu đầu vào xuất hiện sai số trong việc đo lường. Kết quả thu được cho thấy sự ảnh hưởng rõ ràng của sai số từ dữ liệu huấn luyện đến mô hình. Tuy nhiên sau khi đào tạo, độ chính xác của mô hình được đề xuất vẫn giữ được dạng gốc và vẫn có phần vượt trội so với các phương pháp khác tương tự như khi không có sai số trong dữ liệu. 

Kết quả nghiên cứu của chương~\ref{sec:CRB} đã được công bố:
\begin{enumerate}
    \item[] \textbf{Do Hai Son} and Tran Thi Thuy Quynh (2023), ``Impact Analysis of Antenna Array Geometry on Performance of Semi-Blind Structured Channel Estimation for massive MIMO-OFDM systems,'' in \textit{IEEE Statistical Signal Processing Workshop (SSP)}, Hanoi, Vietnam, July. [accepted]
\end{enumerate}