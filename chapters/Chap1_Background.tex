\clearpage
\phantomsection

\setcounter{chapter}{0}
\chapter[TỔNG QUAN VỀ CÁC PHƯƠNG PHÁP NHẬN DẠNG HỆ THỐNG TRONG TRUYỀN THÔNG KHÔNG DÂY]{Tổng quan về các phương pháp nhận dạng hệ thống trong truyền thông không dây}

\begin{figure}[H]
    \centering
    \begin{tikzpicture}
        \node (b1) [startstop] at (0, 0) {Các thuật toán ước lượng kênh truyền};

        \node (b21) [process, align=center] at (-60mm, -25mm) {Sử dụng Pilot};

        \node (b22) [process, align=center] at (-20mm, -25mm) {Thuật toán mù};

        \node (b23) [process, align=center] at (20mm, -25mm) {Thuật toán bán mù};

        \node (b24) [process, align=center] at (60mm, -25mm) {Học máy/Học sâu};

        \node (b31) [below=10mm of b21, process, align=center, fill=green!10!white] {Least Square (LS)~\cite{GesbertSPAWC}};

        \draw[line] (b1.south) -- ([yshift=-5mm]b1.south);
        \draw[arrow] ([yshift=-5mm]b1.south) -| (b21);
        \draw[arrow] ([yshift=-5mm]b1.south) -| (b22);
        \draw[arrow] ([yshift=-5mm]b1.south) -| (b23.north);
        \draw[arrow] ([yshift=-5mm]b1.south) -| (b24.north);

        \draw[arrow] (b21) -- (b31);
        
    \end{tikzpicture}
    \caption{Phân loại các thuật toán ước lượng kênh truyền.}
    \label{fig:classify}
\end{figure}

\section{Các thuật toán nhận dạng kênh tuyến tính}

~\cite{abed1997}

~\cite{Garro2020}

~\cite{bertsekas2014constrained}