\clearpage
\phantomsection

\addcontentsline{toc}{chapter}{KẾT LUẬN}
\chapter*{Kết luận}

Trong luận văn, tác giả tập trung giải quyết các thách thức về chi phí và độ phức tạp của các phương pháp nhận dạng trong các thế hệ mạng di động mới sử dụng mMIMO. Trước hết, một khảo sát về bốn phương pháp nhận dạng hệ thống viễn thông không dây được tác giả trình bày. Qua đó, tác giả chỉ ra sự cần thiết của việc ứng dụng chi thức mới vào bài toán nhận dạng hệ thống thông qua hai hướng tiếp cận là bán mù và sử dụng học sâu. Từ đó, trong chương~\ref{sec:MRE}, tác giả đã đề xuất một phương pháp nhận dạng kênh truyền cho MIMO/mMIMO đó là SB-MRE. Phương pháp sử dụng thông tin từ một số ký tự pilot và tri thức mới từ hướng tiếp cận mù là thuật toán B-MRE trước đó. Ngoài ra, tác giả cũng đề xuất giảm thiểu độ phức tạp của B-MRE nhằm giảm thiểu số lượng bộ cân bằng kênh cần ước lượng. Giải thuật được kiểm chứng qua các mô phỏng cho thấy ưu thế rõ rệt khi chỉ với một số lượng nhỏ pilot, SB-MRE đã tiệm cận và vượt qua ( ở các mức SNR cao) độ chính xác của các phương pháp ZF, MMSE vốn yêu cầu đầy đủ thông tin về kênh truyền. Ngoài ra, đề xuất giảm chi phí của thành phần B-MRE cũng cho thấy kết quả ở mức chấp nhận được đặc biệt khi SNR đủ lớn để thành phần mù bắt đầu cho thấy tác dụng rõ ràng. Trong chương~\ref{sec:ML}, hướng tiếp cận học sâu cũng được tác giả xem xét để nhận dạng kênh truyền cho hệ thống mMIMO. Cụ thể, một mạng ISDNN được đề xuất nhằm giảm thiểu độ phức tạp và chi phí so với thuật toán ISD gốc dựa trên MMSE. Kiến trúc mạng nơ-ron sâu được đề xuất chỉ yêu cầu $24$ tham số học và $7$~KB cho mô hình được đào tạo với cấu hình gồm $4$ lớp mạng. Đây là số lượng rất nhỏ và hoàn toàn vượt trội khi so sánh với một mạng nơ-ron sâu khác cũng với cách tiếp cận tương tự là DetNet. Từ các kết quả mô phỏng, hiệu suất về thời gian đào tạo và độ chính xác của ISDNN cũng được kiểm chứng là vượt trội cả các phương pháp tuyến tính và mạng nơ-ron sâu DetNet. Ngoài ra, tác giả cũng xem xét đến hiệu suất của mạng nếu dữ liệu đầu vào xuất hiện sai số trong việc đo lường. Kết quả thu được cho thấy sự ảnh hưởng rõ ràng của sai số từ dữ liệu huấn luyện đến mô hình. Tuy nhiên sau khi đào tạo, độ chính xác của mô hình được đề xuất vẫn giữ được dạng gốc và vẫn có phần vượt trội so với các phương pháp khác tương tự như khi không có sai số trong dữ liệu. 

Dù đạt được kết quả về mặt mô phỏng và số học, tuy nhiên, vẫn còn các điểm hạn chế có thể được cải thiện trong tương lai của luận văn. Thứ nhất là mô hình kênh truyền trong các giả thiết vẫn ở dạng đơn giản, tương tự như kênh truyền của mMIMO ở trạng thái lý tưởng khi kênh cứng. Tuy nhiên, trên thực tế, các mô hình kênh 2D, 3D có thể cho độ lợi về độ chính xác và chi phí~\cite{shaik2021}, do vậy tác giả có thể xem xét đến việc thay đổi mô hình kênh truyền trong tương lai. Thứ hai, các phương pháp như ISDNN hay DetNet được đề xuất hiện chỉ được thử nghiệm hoạt động với các kênh truyền có phân bố xác định. Tuy nhiên, điều này ít xảy ra trên thực tế, việc cải tiến, kiểm nghiệm mô hình ISDNN để đáp ứng cho trường hợp kênh truyền có phân bố thay đổi liên tục là cần thiết để đề xuất có tính ứng dụng thực tiễn hơn nữa.