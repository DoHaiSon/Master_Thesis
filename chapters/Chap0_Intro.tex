\clearpage
\phantomsection

\addcontentsline{toc}{chapter}{{MỞ ĐẦU}}
\chapter*{Mở đầu}
\noindent{\Large \textbf{Lý do chọn đề tài}}


\vspace{0.5cm}

\noindent{\Large \textbf{Phương pháp nghiên cứu}}

Cụ thể các phương pháp nghiên cứu sau đã được sử dụng trong luận văn:

\renewcommand{\labelitemi}{$-$}
\begin{itemize}
	\item 
	\item 
	\item 
	\item 
\end{itemize} 
\vspace{0.3cm}

\noindent{\Large \textbf{Nội dung nghiên cứu}}

\renewcommand{\labelitemi}{$-$}
\begin{itemize}
	\item 
	\item 
	\item 
	\item 
	\item 
\end{itemize} 
\vspace{0.3cm}

\noindent{\Large \textbf{Đóng góp của đề tài}}

Với sự hiểu biết của học viên, những kết quả nghiên cứu trong luận văn đã đạt được mục đích nghiên cứu đề ra. Những kết quả này bao gồm:

\renewcommand{\labelitemi}{$-$}
\begin{itemize}
	\item 
	\item 
	\item 
	\item
	\item 
\end{itemize} 

\noindent{\Large \textbf{Bố cục của luận văn}}
\vspace{0.5cm}

Nội dung chính của luận văn được trình bày như sau:

\renewcommand{\labelitemi}{$-$}
\begin{itemize}
	\item Mở đầu: Trình bày mục đích, phương pháp nghiên cứu, nội dung, đóng góp và bố cục của luận văn.
	\item Chương 1: 
	\item Chương 2: 
	\item Chương 3: 
	\item 
\end{itemize} 